\begin{cabstract}
  \renewcommand{\chapterlabel}{摘\hspace{2em}要}

  经典线性回归模型是最简单的回归模型,用来研究某些变量对另一变量的线性影响,但其
  有较多严格的假设条件。在实际应用中,数据不一定满足全部的假设条件。当数据违背经
  典线性回归模型的假设时,继续使用经典线性回归模型是不合适的。这时可以引入新的模
  型来解决违背经典线性回归模型假设的情况。广义线性模型的引入解决了离散因变量的问
  题,广义可加模型的引入解决了自变量所带来的非线性影响。在实际研究中,针对相关数
  据,可以通过在广义可加模型中引入随机效应来反映这些异质性,这就是广义可加混合模
  型(GAMM, Generalized Additive Mixed Model)。
  
  广义可加混合模型具有很大的灵活性,但其参数估计方法较为复杂,这在某种程度限制了
  其实际应用。随着广义可加混合模型参数估计方法的发展及统计软件对这些方法的实现,
  快速估计广义可加混合模型已成为可能。本项目考虑GAMM模型在一个重要环境问题上的应
  用:空气质量问题。本项目选取 2014-2017 年中国重点监测的35 个城市的年度和月度数
  据建立广义可加混合模型进行相应研究。除把空气质量指数作为因变量分析外,还借助于
  广义可加混合模型的广泛适用性,研究了空气质量为优天数和出现空气严重污染好坏两种
  极端情况,使得对于空气质量的研究更为完整。
  
  研究报告在梳理国内外有关广义可加混合模型及空气质量影响因素有关文献的基础上,给
  出了广义可加混合模型的模型背景和模型设定,讨论了广义可加混合模型的两种参数估计
  方法:DPQL估计和基于切片抽样MCMC估计。在城市空气质量实证部分,数据选取范围为重
  点监测的35个城市,收集了2014-2017年的年度和月度的空气质量指数及可能的影响因素变
  量数据,包括空经济社会因素,气象因素和地理因素共24个影响变量。利用收集到的数据
  进行城市空气质量状况描述,并对空气质量指数的时空分布特征进行分析,通过计算全局
  空间自相关指数,讨论了空气质量指数的时空格局演变。最后项目分别从空气质量指数,
  空气质量为优天数和空气严重污染三个不同角度,建立各种广义可加混合模型并通过模型
  比较和选择研究影响空气质量的因素。项目通过以上研究主要得到以下几点结论:
  
  1.项目对城市空气质量状况进行了基本描述并分析了其时空分布特征,主要分析
  了2017年35个城市的空气质量的基本概况及其与社会经济影响因素,气象因素和地理因
  素3个方面24个自变量之间的关系。从时间角度看,整体而言我国大部分城市空气质量还是
  有随时间逐渐好转的趋势,但少数城市空气质量有逐年恶化的趋势,如太原,西安,另有
  部分城市空气质量指数持续处于高位状态,如石家庄。空间分析结果发现:以秦岭淮河为
  分界线,整体来说,北方城市的空气质量较差,南方城市的空气质量相对较好,且地理位
  置越靠近南方,空气质量越好。以北京、天津、石家庄、太原、济南、西安为代表的城市
  空气质量较差,且其存在显著的空间正自相关,以海口、贵阳、昆明、南宁、福州、厦门
  为代表的城市空气质量较好,存在显著的空间负相关。
  
  2.在建模过程中,我们考察的因变量一共有 6 个,分别为年均空气质量指数,月均空气质
  量指数,年度空气质量为优天数,月度空气质量为优天数,年度空气严重污染和月度空气
  严重污染。对这六个变量分别采用 Shapiro-Wilk 检验和 Jarque- Bera检验进行正态性检
  验,通过对检验结果分析并考虑数据的特征,最终确定选取的因变量服从的分布为:年均
  空气质量指数(正态分布),年度空气质量为优的天数(取对数后近似正态分布)、年度
  严重污染的天数(泊松分布);月均空气质量指数(取对数后近似正态分布),月度空气
  质量为优的天数(泊松分布)、月度是否发生空气严重污染(二点分布)。
  
  3.对于年均空气质量指数,通过建立正态可加混合模型并进行模型比较和选择,最优模型
  揭示社会经济影响因素中有废气治理完成投资额,固定资产投资额和汽车拥有量,其中废
  气治理完成投资额是正的线性影响,固定资产投资和汽车拥有量与年均空气质量指数之间
  具有非线性关系;气候因素中的平均风速和平均气温也对年均空气质量指数有显著性影响,
  其中平均风速是线性关系,平均气温具有非线性关系。是否临海,是否供暖和纬度三个地
  理因素对年均空气质量指数有线性影响。对于月均空气质量指数,对其取对数后并建立正
  态可加混合模型并进行模型比较和选择,最优模型揭示社会经济方面影响因素有废气治理
  完成投资额,房屋建筑施工面积和汽车拥有量,且这三个因素均为线性影响;平均气温,
  降水量,平均风速和平均气压这四个气象变量都对月均空气质量指数有显著性的线性影响;
  是否临海,是否供暖和纬度三个地理因素对月均空气质量指数有影响其中是否临海和是否
  供暖是线性影响,纬度与月均空气质量指数直接的关系是非线性的。
  
  4.对于年度空气质量为优天数,对其取对数后建立正态可加混合模型并进行模型比较和选
  择,最优模型揭示社会经济影响因素有GDP,全社会用电量和废气治理完成投资额,其中全
  社会用电量和废气治理完成投资额与年度空气质量为优天数之间具有线性关系,GDP具有非
  线性作用;气候因素中的平均相对湿度和平均风速对年度空气质量为优天数有线性影响。
  地理因素中只有纬度变量对年度空气质量为优天数有显著性影响,并且关系是非线性的。
  对于月度空气质量为优天数,通过建立泊松可加混合模型并进行模型比较和选择,最优模
  型揭示社会经济影响因素有房屋建筑施工面积和废气治理完成投资额,并且二者与年度空
  气质量为优天数之间均具有线性关系;气候因素中的平均气温,降水量,平均相对湿度和
  平均风速对月度空气质量为优天数有线性影响。地理因素中是否供暖和纬度变量对月度空
  气质量为优天数有显著性影响,是否供暖具有线性关系,纬度变量具有非线性影响。
  
  5.对于年度空气严重污染天数,通过建立泊松可加混合模型并进行模型比较和选择,最优
  模型揭示社会经济影响因素中的常住人口数是影响年度空气严重污染天数多少的主要因素,
  并且这个影响非线性的。总体而言,年度空气严重污染天数是随着常住人口数的增加而逐
  渐增多,但常住人口数超过2000万的特大城市,对于空气严重污染控制的比较好,年度空
  气严重污染天数并不多。气候因素中的降水量和平均风速对年度空气严重污染天数有负向
  线性影响,而日照时数有非线性影响。地理因素中的是否供暖,是否临海和纬度均对年度
  空气严重污染天数有显著性线性影响。对于月度是否出现空气严重污染,通过建
  立Logistic可加混合模型并进行模型比较和选择,最优模型揭示社会经济影响因素有常住
  人口数和私人汽车拥有量,其中私人汽车拥有量与月度是否出现空气严重污染的概率之间
  具有负向线性关系,常住人口数与月度是否出现空气严重污染概率之间具有非线性关系;
  气候因素中的平均气温,降水量,平均相对湿度,平均风速和日照时数对月度是否出现空
  气严重污染的概率局有显著性影响,其中平均气温是非线性影响,其他4个气候因素是负向
  线性影响。地理因素中只有纬度变量对月度是否出现空气严重污染有显著性影响,并且这
  个影响是非线性的。
  
  \bigbreak

  {\bfseries 关键词}:广义可加混合模型;双重惩罚拟似然估计;切片抽样;MCMC估计;空气质量指数;影响因素
   

\end{cabstract}





